\documentclass[titlepage, a4paper, 12pt]{article}
\usepackage[swedish]{babel}
\usepackage[utf8]{inputenc}
\usepackage{verbatim}
\usepackage{fancyhdr}
\usepackage{graphicx}
\usepackage{parskip}

% SourceCode
\usepackage{listings}
\usepackage{color}

% Include pdf with multiple pages ex \includepdf[pages=-, nup=2x2]{filename.pdf}
\usepackage[final]{pdfpages}
% Place figures where they should be
\usepackage{float}

% SourceCode
\definecolor{keywordcolor}{rgb}{0.5,0,0.75}
\lstset{
  inputencoding=utf8,
  language=Java,
  extendedchars=true,
  basicstyle=\scriptsize\ttfamily,
  stringstyle=\color{blue},
  commentstyle=\color{red},
  numbers=left,
  firstnumber=auto,
  numberblanklines=true,
  stepnumber=1,
  showstringspaces=false,
  keywordstyle=\color{keywordcolor}
  % identifierstyle=\color{identifiercolor}
}

% Float for text
\floatstyle{ruled}
\newfloat{kod}{H}{lop}
\floatname{kod}{Kodsnutt}

% vars
\def\title{Termiter i NetLogo}
\def\preTitle{Laboration 1}
\def\kurs{Emergenta system, VT-09}

\def\namn{Andreas Jakobsson}
\def\mail{dit06ajs@cs.umu.se}
\def\namnTva{Anton Johansson}
\def\mailTva{dit06ajn@cs.umu.se}

\def\pathtocode{$\sim$dit06ajn/edu/emergenta-system/lab1/src}

\def\handledareEtt{Jonny Pettersson, jonny@cs.umu.se}
\def\handledareTva{Anders Broberg, bopspe@cs.umu.se}

\def\inst{datavetenskap}
\def\dokumentTyp{Laborationsrapport}

\begin{document}
\begin{titlepage}
  \thispagestyle{empty}
  \begin{small}
    \begin{tabular}{@{}p{\textwidth}@{}}
      UMEÅ UNIVERSITET \hfill \today \\
      Institutionen för \inst \\
      \dokumentTyp \\
    \end{tabular}
  \end{small}
  \vspace{10mm}
  \begin{center}
    \LARGE{\preTitle} \\
    \huge{\textbf{\kurs}} \\
    \vspace{10mm}
    \LARGE{\title} \\
    \vspace{15mm}
    \begin{large}
      \namn, \mail \\
      \namnTva, \mailTva\\
      \texttt{\pathtocode}
    \end{large}
    \vfill
    \large{\textbf{Handledare}}\\
    \mbox{\large{\handledareEtt}}
    \mbox{\large{\handledareTva}}
  \end{center}
\end{titlepage}

\newpage
\mbox{}
\vspace{70mm}
\begin{center}
% Dedication goes here
\end{center}
\thispagestyle{empty}
\newpage

\pagestyle{fancy}
\rhead{\today}
\lhead{\footnotesize{\namn, \mail\\\namnTva, \mailTva}}
\chead{}
\lfoot{}
\cfoot{}
\rfoot{}

\cleardoublepage
\newpage
\tableofcontents
\cleardoublepage

\fancyfoot[LE,RO]{\thepage}
\pagenumbering{arabic}

\section{Problemspecifikation}\label{sec:problemspecifikation}
Laborationen gick ut på att göra ändringar i en befintlig
NetLogo\footnote{http://ccl.northwestern.edu/netlogo/} modell som
imiterar myrors beteende att bygga myrstackar. Ändringarna som skulle
göras och studeras var:

\begin{itemize}
\item Sortering av flera sorters träbitar för att bygga olika
  myrstackar.
\item Laborera med feltolerans, vad händer med modellen om det införs
  myror som på något sätt stör de andra myrornas beteende?
\end{itemize}
% Behövs det ytterligare regler i termiterna för att systemet ska
% konvergera till en separat hög för varje sorts träbitar?

Laborationsspecifikation finns i original på sidan:\\
\verb!http://www.cs.umu.se/kurser/5DV017/VT09/lab/lab1.html!

\section{Analys}
Nedan avsnitt beskriver analysen som gjorts med avseende på frågorna
från problemspecifikationen.

\subsection{Sortering av olika träbitar}
% Den ursprungliga termitmodellen innehåller endast en sorts termiter
% och en sorts träbitar. Utöka modellen till minst två sorters termiter
% och två sorters träbitar. Behövs det ytterligare regler i termiterna
% för att systemet ska konvergera till en separat hög för varje sorts
% träbitar? Behövs det ytterligare regler för att minska
% konvergeringstiden? Med konvergering menas här att det endast finns en
% hög av varje sorts träbitar när systemet konvergerat.


\subsection{Feltolerans}
% I ett system är feltolerans en viktig egenskap, ett naturligt eller
% artificiellt system måste tåla i en viss utsträckning att saker går
% fel och att andra agenter i dess omgivning vill förstöra. Studera hur
% termitmodellen reagerar om det införs termiter som på något sätt
% förstör (exempelvis genom att plocka upp träbitar och lägga ned dem
% där inga andra träbitar finns). Frågor att fundera kring är: Hur stor
% andel förstörande termiter behövs det minst för att förstöra
% uppbyggnaden av en hög? Hur påverkas konvergeringshastigheten? Studera
% detta med endast en sorts termiter/träbitar och med flera sorters
% termiter/träbitar.

%  I er inlämnade lösning ska det enkelt gå att ändra antalet av
%  respektive sort termiter och andelen av respektive sort träbitar.

\bibliographystyle{alpha}
\bibliography{books.bib}

\newpage
\appendix
\pagenumbering{roman}
\section{Bilagor}\label{sec:kallkod}
% Källkoden ska finnas tillgänglig i er hemkatalog
% ~/edu/apjava/lab1/. Bifoga även utskriven källkod.
Härefter följer utskrifter från källkoden och andra filer som hör till
denna laboration

\subsection{Källkod}
\subsection{Termites.nlogo}\label{Termites.nlogo}
\begin{footnotesize}
  \verbatiminput{../src/Termites.nlogo}
\end{footnotesize}
\end{document}
