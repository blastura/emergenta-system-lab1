\documentclass[titlepage, a4paper, 12pt]{article}
\usepackage[swedish]{babel}
\usepackage[utf8]{inputenc}
\usepackage{verbatim}
\usepackage{fancyhdr}
\usepackage{graphicx}
\usepackage{parskip}

% SourceCode
\usepackage{listings}
\usepackage{color}

% Include pdf with multiple pages ex \includepdf[pages=-, nup=2x2]{filename.pdf}
\usepackage[final]{pdfpages}
% Place figures where they should be
\usepackage{float}

% SourceCode
\definecolor{keywordcolor}{rgb}{0.5,0,0.75}
\lstset{
  inputencoding=utf8,
  language=Java,
  extendedchars=true,
  basicstyle=\scriptsize\ttfamily,
  stringstyle=\color{blue},
  commentstyle=\color{red},
  numbers=left,
  firstnumber=auto,
  numberblanklines=true,
  stepnumber=1,
  showstringspaces=false,
  keywordstyle=\color{keywordcolor}
  % identifierstyle=\color{identifiercolor}
}

% Float for text
\floatstyle{ruled}
\newfloat{kod}{H}{lop}
\floatname{kod}{Kodsnutt}

% vars
\def\title{Termiter i NetLogo}
\def\preTitle{Laboration 1}
\def\kurs{Emergenta system, VT-09}

\def\namn{Andreas Jakobsson}
\def\mail{dit06ajs@cs.umu.se}
\def\namnTva{Anton Johansson}
\def\mailTva{dit06ajn@cs.umu.se}

\def\pathtocode{$\sim$dit06ajn/edu/emergenta-system/lab1/src}

\def\handledareEtt{Jonny Pettersson, jonny@cs.umu.se}
\def\handledareTva{Anders Broberg, bopspe@cs.umu.se}

\def\inst{datavetenskap}
\def\dokumentTyp{Laborationsrapport}

\begin{document}
\begin{titlepage}
  \thispagestyle{empty}
  \begin{small}
    \begin{tabular}{@{}p{\textwidth}@{}}
      UMEÅ UNIVERSITET \hfill \today \\
      Institutionen för \inst \\
      \dokumentTyp \\
    \end{tabular}
  \end{small}
  \vspace{10mm}
  \begin{center}
    \LARGE{\preTitle} \\
    \huge{\textbf{\kurs}} \\
    \vspace{10mm}
    \LARGE{\title} \\
    \vspace{15mm}
    \begin{large}
      \namn, \mail \\
      \namnTva, \mailTva\\
      \texttt{\pathtocode}
    \end{large}
    \vfill
    \large{\textbf{Handledare}}\\
    \mbox{\large{\handledareEtt}}
    \mbox{\large{\handledareTva}}
  \end{center}
\end{titlepage}

\newpage
\mbox{}
\vspace{70mm}
\begin{center}
% Dedication goes here
\end{center}
\thispagestyle{empty}
\newpage

\pagestyle{fancy}
\rhead{\today}
\lhead{\footnotesize{\namn, \mail\\\namnTva, \mailTva}}
\chead{}
\lfoot{}
\cfoot{}
\rfoot{}

\cleardoublepage
\newpage
\tableofcontents
\cleardoublepage

% \fancyfoot[LE,RO]{\thepage}
\cfoot{\thepage}
\pagenumbering{arabic}


\section{Inledning}
Denna laboration ska undersöka den emergenta egenskapen som uppstår
med myror som var och en följer följande enkla regler:

\begin{itemize}
 \item Vandra slumpmässigt omkring (utan bärande på träbit)
 \item Vid kontakt med en träbit, plocka upp den
 \item Vandra slumpmässigt omkring (bärande på träbit)
 \item Vid kontakt med en träbit, lägg ned den egna träbiten brevid den nya
 \item Hoppa till första punkten igen
\end{itemize}

Med hjälp av dessa enkla regler kan det i interaktion mellan myror
(eller en myra med sig själv över tid) uppträda den globala egenskapen
av en samlad hög.

\section{Problemspecifikation}\label{sec:problemspecifikation}
Laborationen gick ut på att göra ändringar i en befintlig
NetLogo\footnote{http://ccl.northwestern.edu/netlogo/} modell som
imiterar myrors beteende att bygga myrstackar. Ändringarna som skulle
göras och studeras var:

\begin{itemize}
\item Sortering av flera sorters träbitar för att bygga olika
  myrstackar.
\item Laborera med feltolerans, vad händer med modellen om det införs
  myror som på något sätt stör de andra myrornas beteende?
\end{itemize}
% Behövs det ytterligare regler i termiterna för att systemet ska
% konvergera till en separat hög för varje sorts träbitar?

\subsection{Frågor denna rapport behandlar}
Följande lista innehåller frågor som denna laboration diskuterar
kring:

\begin{itemize}
\item Behövs det ytterligare regler i termiterna för att systemet ska
  konvergera till en separat hög för varje sorts träbitar?
\item Behövs det ytterligare regler för att minska konvergeringstiden?
  (Med konvergering menas här att det endast finns en hög av varje
  sorts träbitar när systemet konvergerat.)
\item Hur stor andel förstörande termiter behövs det minst för att
  förstöra uppbyggnaden av en hög? (hur definieras förstöra?)
\item Hur påverkar förstörarmyrorna konvergeringshastigheten?
  % TODO: se nedan.
\item OBS Studera detta med endast en sorts termiter/träbitar och med
  flera sorters termiter/träbitar.
\item OBS I er inlämnade lösning ska det enkelt gå att ändra antalet
  av respektive sort termiter och andelen av respektive sort träbitar.
\item Kan du se några applikationsområden för den här typen av algoritmer?
\item Kan modellen utökas för att göras mer intressant?
\end{itemize}

Laborationsspecifikation finns i original på sidan:\\
\verb!http://www.cs.umu.se/kurser/5DV017/VT09/lab/lab1.html!

\section{Användarhandledning}
Källkoden till implementationen Termites.nlogo som diskuteras i denna
rapport finns att hitta på:

\verb!~dit06ajn/edu/emergenta-system/lab1/src!

Öppna filen i NetLogo för att köra den.

\subsection{Förklaring av användargränssnittet}
Nedan följer en förklaring av de knappar, reglage och monitorer som
förekommer i användargränssnittet:

\begin{itemize}
\item \textbf{session-name} - används till filnamn för skärmdumpar.
\item \textbf{yellowAnts} - antalet myror som intresserar sig av gula träbitar.
\item \textbf{greenAnts} - antalet myror som intresserar sig av gröna träbitar.
\item \textbf{number-of-bad-ants} - antalet myror som saboterar högbildandet.
\item \textbf{density} - andelen av världens rutor som ska
  representera träbitar. Denna implementationen placerar ut hälfen gröna
  och hälften gula träbitar. Ex. 20 \% ger 10 \% gröna och 10 \% gula)
\item \textbf{setup} - denna knapp initierar parametrarna till världen.
\item \textbf{go} - denna knapp sätter igång animeringen i
  världen. Setup måste köras en gång innan denna knapp får användas.
\item \textbf{save-snapshot} - denna knapp sparar en skärmdump av
  gränssnittet i katalogen src/testruns.
\item \textbf{good-bad-ratio} - denna monitor visar kvoten av "snälla"
  och "onda" myror.
\item \textbf{yellow-green-ratio} - denna monitor visar kvoten av gula
  och gröna myror.
\item \textbf{two-breeds?} - ställer in om det ska finnas två olika
  sorters träbitar och myror.
\item \textbf{Run test script} - denna knapp startar en metod som kör
  ett antal hårdkodade test. Skärmdumpar kommer sparas under
  körningens gång i katalogen src/testruns
\item \textbf{fillness} - detta diagram visar fillness-ratio.
\item \textbf{fillness-ratio} - denna monitor visar andelen träbitar där
  alla grannar är av samma färg. Variabeln uppdateras var tioende
  tick.
\item \textbf{max-fillness-ratio} - denna monitor visar globala maxima
  för variabeln fillness-ratio
\end{itemize}

\section{Metod för testning}\label{sec:metod-for-testning}
% Test ants 100-100, density 20 %, variera badants 0, 10, 20, 30, 40,
% max ticks väljer vi till 2000 eftersom det gott och väl räcker för
% konvergering utan förstörande element.

Inledningvis användes en trial-and-error metodik för att testköra
implementationen och tolka vad de olika parametrarna har för inverkan
i simuleringen. För att sedan generera mätbar testdata implementerades
en metod som räknade antalet träbitar där alla grannar var av samma
färg och därefter beräkna en kvot av dessa träbitar dividerat med
totalt antal träbitar. Detta blir en fingervisning på hur långt
konvergensen har nått. Maximal kvot uppstår när träbitarna ligger
samlade i en boll då bollen har maximal area mot omkrets, alltså minst
träbitar i ytterkant.

För att skapa testfall skrevs en metod som körde ett antal
testkörningar med olika parameterinställningar och lagrade dessa genom
att var 500:ade \textit{tick} spara en skärmdump, och vid var 10:onde
\textit{tick} spara antalet \textit{ticks} och aktuell
konvergeringskvot.

Testkörningarna kördes maximalt 2000 \textit{ticks}, vilket gott och
väl räckte för konvergering av högar i testfall där få förstörande
myror fanns med.

% TODO fyll på spara data o grafer

\section{Reflektioner}
Nedan avsnitt beskriver analysen som gjorts med avseende på frågorna
från problemspecifikationen.

\subsection{Sortering av olika träbitar}
% Reflektion kring er lösning och eventuella begränsningar
% Den ursprungliga termitmodellen innehåller endast en sorts termiter
% och en sorts träbitar. Utöka modellen till minst två sorters termiter
% och två sorters träbitar. Behövs det ytterligare regler i termiterna
% för att systemet ska konvergera till en separat hög för varje sorts
% träbitar? Behövs det ytterligare regler för att minska
% konvergeringstiden? Med konvergering menas här att det endast finns en
% hög av varje sorts träbitar när systemet konvergerat.
För implementera sortering av olika träbitar gjordes två försök. Ett
med en myrsort och fler regler än de som fanns i den ursprungliga
modellen, samt ett försök med två myrsorter liknande den i den givna
modellen.

Vid försök med att låta en och samma myrsort plocka upp olika
träsorter men enbart lägga ner dem när de hittat en träsort av typ,
uppstod problem med att högar bildades av blandade träsorter. Detta
eftersom myrorna aldrig gick in mot mitten av en hög för att plocka
träbitar. Så en myra hittade en träbit plockades den upp och lades ned
vid första kontakt med liknande träbit. Detta resulterade i att
myrorna alltid höll sig i utkanten av högarna. Ingen sortering av
redan skapade högar utfördes.

Vid försök med olika myrsorter som var för sig enbart plockade upp och
lade ner träbitar av en specifik sort, konvergerade högarna
separat. % TODO, se figur \ref}.
Det krävdes alltså inga nya regler utan bara nya myr- och
träsorter.

För att minska tiden det tog för myrorna att av varje träsort enbart
bilda en samlad hög ökades stegen som myrorna tog mellan varje
procedur. Denna ändring gjordes i metoden \textit{wiggle}, steglängden
ökades från ett steg till tjugo.
% TODO reflektioner över detta? Tester osv.

\subsection{Feltolerans}
% I ett system är feltolerans en viktig egenskap, ett naturligt eller
% artificiellt system måste tåla i en viss utsträckning att saker går
% fel och att andra agenter i dess omgivning vill förstöra. Studera hur
% termitmodellen reagerar om det införs termiter som på något sätt
% förstör (exempelvis genom att plocka upp träbitar och lägga ned dem
% där inga andra träbitar finns). Frågor att fundera kring är: Hur stor
% andel förstörande termiter behövs det minst för att förstöra
% uppbyggnaden av en hög? Hur påverkas konvergeringshastigheten? Studera
% detta med endast en sorts termiter/träbitar och med flera sorters
% termiter/träbitar.

%  I er inlämnade lösning ska det enkelt gå att ändra antalet av
%  respektive sort termiter och andelen av respektive sort träbitar.

För att testa feltoleransen i systemet lades myror in som plockar upp
alla träsorter de stöter på och efter ett bestämt antal steg försöker
lägga ner denna träbit på en ledig plats.

Test gjordes där kvoten mellan jobbande myror och förstörande myror
varierades. Antalet förstörande myror har varierats mellan 0 och 50 stycken,
antalet arbetande myror har varit 100 stycken av varje sort och med
densitet av 10 \% per träsort. Efter som modellen bygger på
slumpfaktorer, till exempel myrornas riktning och träbitarnas
startpositioner, har varje test har körts fem gånger för att förhindra
skev bild av förloppen på grund av slumpen. 

Träbitarnas fyllnadsgrad, se avsnitt \ref{sec:metod-for-testning}, har
plottats i förhållande till antal \textit{ticks} som körts i figur
\ref{fig:images/graph0} till \ref{fig:images/graph50oneBreed}.



En krans av spridda träbitar runt
högarna tyder på den kontinuerliga förstörelsen.

Vi antar att det kommer att krävas fler antal förstörande myror än
antalet uppbyggande myror för att hindra konvergering av högarna. Ett
test av detta skulle ta allt för lång tid att genomföra.

% TODO gamla bilder borttagna

% Grafer
\begin{figure}
  \begin{minipage}[b]{0.5\linewidth}
    \centering
    \caption{0  bad ants }\label{fig:images/graph0}
    \includegraphics[width=6cm]{images/graph0.pdf}
  \end{minipage}
  \begin{minipage}[b]{0.5\linewidth}
    \centering
    \caption{0 bad ants, one breed }\label{fig:images/graph0oneBreed}
    \includegraphics[width=6cm]{images/graph0oneBreed.pdf}
  \end{minipage}
  
  \hspace{0.5cm}
  
  \begin{minipage}[b]{0.5\linewidth}
    \centering
    \caption{10 bad ants }\label{fig:images/graph10}
    \includegraphics[width=6cm]{images/graph10.pdf}
  \end{minipage}
  \begin{minipage}[b]{0.5\linewidth}
    \centering
    \caption{10 bad ants, one breed }\label{fig:images/graph10oneBreed}
    \includegraphics[width=6cm]{images/graph10oneBreed.pdf}
  \end{minipage}
  
  \hspace{0.5cm}

  \begin{minipage}[b]{0.5\linewidth}
    \centering
    \caption{20 bad ants }\label{fig:images/graph20}
    \includegraphics[width=6cm]{images/graph20.pdf}
  \end{minipage}
  \begin{minipage}[b]{0.5\linewidth}
    \centering
    \caption{20 bad ants, one breed }\label{fig:images/graph20oneBreed}
    \includegraphics[width=6cm]{images/graph20oneBreed.pdf}
  \end{minipage}
  
  \hspace{0.5cm}

  \begin{minipage}[b]{0.5\linewidth}
    \centering
    \caption{30 bad ants }\label{fig:images/graph30}
    \includegraphics[width=6cm]{images/graph30.pdf}
  \end{minipage}
  \begin{minipage}[b]{0.5\linewidth}
    \centering
    \caption{30 bad ants, one breed }\label{fig:images/graph30oneBreed}
    \includegraphics[width=6cm]{images/graph30oneBreed.pdf}
  \end{minipage}
  
\end{figure}

\begin{figure}

  \begin{minipage}[b]{0.5\linewidth}
    \centering
    \caption{40 bad ants }\label{fig:images/graph40}
    \includegraphics[width=6cm]{images/graph40.pdf}
  \end{minipage}
  \begin{minipage}[b]{0.5\linewidth}
    \centering
    \caption{40 bad ants, one breed }\label{fig:images/graph40oneBreed}
    \includegraphics[width=6cm]{images/graph40oneBreed.pdf}
  \end{minipage}
  
  \hspace{0.5cm}
  
  \begin{minipage}[b]{0.5\linewidth}
    \centering
    \caption{50 bad ants }\label{fig:images/graph50}
    \includegraphics[width=6cm]{images/graph50.pdf}
  \end{minipage}
  \begin{minipage}[b]{0.5\linewidth}
    \centering
    \caption{50 bad ants, one breed }\label{fig:images/graph50oneBreed}
    \includegraphics[width=6cm]{images/graph50oneBreed.pdf}
  \end{minipage}
\end{figure}
                    
\section{Applikationsområden}
% Diskutera de frågor som nämns ovan och försöka att sätta in
% laborationen i ett större sammanhang. Kan du se några
% applikationsområden för den här typen av algoritmer? Kan modellen
% utökas för att göras mer intressant?

Tänkta applikationer för denna typen av algoritmer kan vara till
exempel olika mediciner och behandlingar på nano-nivå där mycket enkla
små agenter gemensamt kan konvergera till en komplex behandlingsform.

Kluster av billiga datorer kan jobba gemensamt för att bilda ett
komplext beteende. Distribuerad bearbetning och lagring av data på
datorer runt om världen. Exempel på sådana projekt är exempel
lösningar som räknar fram stora primtal eller behandlar signaler och
letar efter tecken på utomjordiskt liv.

\section{Lösningens begränsningar}
Om ett densitetsreglage fanns till varje färg hade vi problem med att
verkligen få rätt densitet till varje färg. När den första färgen har
slumpats ut vill vi slumpa ut nästa färg. Här misslyckades vi med att
lägga den nya färgen på enbart lediga rutor och alltså kunde vi måla
över den gamla färgen. Detta löste vi genom att ha ett reglage med
densitet och för varje slumpad plats målas med sannolikheten 0.5 den
gröna färgen och med sannolikheten 0.5 den gula färgen. Att ha ett
reglage till varje färg är abolut önskvärt för testning och detta
skulle vara en bra grej att fixa om mer tid fanns.

När vi gjorde om modellen för att lägga in en \textit{ticks}-räknare
var vi tvungna att ändra \textit{go}-proceduren så att den körs av en
\textit{observer}. Ändringarna fick som sidoeffekt att varje myra kör
varje procedur sekvensiellt istället för att alla myror agerar
samtidigt. Vi antar dock att resultaten blir intressant ändå i och med
att en myra bara kollar på pinnar och inte på andra myrors
rörelser. % TODO (STÄMMET  DETTA?  ?

\newpage
\appendix
\pagenumbering{roman}
\section{Källkod}\label{sec:kallkod}
% Källkoden ska finnas tillgänglig i er hemkatalog
% ~/edu/apjava/lab1/. Bifoga även utskriven källkod.
Härefter följer utskrifter från källkoden och andra filer som hör till
denna laboration

\subsection{Termites.nlogo}\label{Termites.nlogo}
\begin{footnotesize}
  \verbatiminput{../src/Termites.nlogo}
\end{footnotesize}
\end{document}
