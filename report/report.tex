\documentclass[titlepage, a4paper, 12pt]{article}
\usepackage[swedish]{babel}
\usepackage[utf8]{inputenc}
\usepackage{verbatim}
\usepackage{fancyhdr}
\usepackage{graphicx}
\usepackage{parskip}

% SourceCode
\usepackage{listings}
\usepackage{color}

% Include pdf with multiple pages ex \includepdf[pages=-, nup=2x2]{filename.pdf}
\usepackage[final]{pdfpages}
% Place figures where they should be
\usepackage{float}

% SourceCode
\definecolor{keywordcolor}{rgb}{0.5,0,0.75}
\lstset{
  inputencoding=utf8,
  language=Java,
  extendedchars=true,
  basicstyle=\scriptsize\ttfamily,
  stringstyle=\color{blue},
  commentstyle=\color{red},
  numbers=left,
  firstnumber=auto,
  numberblanklines=true,
  stepnumber=1,
  showstringspaces=false,
  keywordstyle=\color{keywordcolor}
  % identifierstyle=\color{identifiercolor}
}

% Float for text
\floatstyle{ruled}
\newfloat{kod}{H}{lop}
\floatname{kod}{Kodsnutt}

% vars
\def\title{AntiTD}
\def\preTitle{Laboration 4}
\def\kurs{Applikationsprogrammering i Java, HT-08}

\def\namn{Andreas Jakobsson}
\def\mail{dit06ajs@cs.umu.se}
\def\namnTva{Anton Johansson}
\def\mailTva{dit06ajn@cs.umu.se}

\def\pathtocode{$\sim$dit06ajn/edu/apjava/lab4}

\def\handledareEtt{Johan Eliasson, johane@cs.umu.se}
\def\handledareTva{Tor Sterner-Johansson, tors@cs.umu.se}

\def\inst{datavetenskap}
\def\dokumentTyp{Laborationsrapport}

\begin{document}
\begin{titlepage}
  \thispagestyle{empty}
  \begin{small}
    \begin{tabular}{@{}p{\textwidth}@{}}
      UMEÅ UNIVERSITET \hfill \today \\
      Institutionen för \inst \\
      \dokumentTyp \\
    \end{tabular}
  \end{small}
  \vspace{10mm}
  \begin{center}
    \LARGE{\preTitle} \\
    \huge{\textbf{\kurs}} \\
    \vspace{10mm}
    \LARGE{\title} \\
    \vspace{15mm}
    \begin{large}
      \namn, \mail \\
      \namnTva, \mailTva\\
      \texttt{\pathtocode}
    \end{large}
    \vfill
    \large{\textbf{Handledare}}\\
    \mbox{\large{\handledareEtt}}
    \mbox{\large{\handledareTva}}
  \end{center}
\end{titlepage}

\newpage
\mbox{}
\vspace{70mm}
\begin{center}
% Dedication goes here
\end{center}
\thispagestyle{empty}
\newpage

\pagestyle{fancy}
\rhead{\today}
\lhead{\footnotesize{\namn, \mail\\\namnTva, \mailTva}}
\chead{}
\lfoot{}
\cfoot{}
\rfoot{}

\cleardoublepage
\newpage
\tableofcontents
\cleardoublepage

\fancyfoot[LE,RO]{\thepage}
\pagenumbering{arabic}

\section{Problemspecifikation}\label{sec:problemspecifikation}

\bibliographystyle{alpha}
\bibliography{books.bib}

\newpage
\appendix
\pagenumbering{roman}
\section{Bilagor}\label{sec:kallkod}
% Källkoden ska finnas tillgänglig i er hemkatalog
% ~/edu/apjava/lab1/. Bifoga även utskriven källkod.
Härefter följer utskrifter från källkoden och andra filer som hör till
denna laboration

\subsection{Källkod}
\subsection{Redovisning av inblandning}

\end{document}
